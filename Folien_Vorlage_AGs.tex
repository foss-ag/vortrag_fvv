\documentclass[xcolor=table]{beamer}

\setbeamercovered{transparent}

\usepackage[utf8]{inputenc}
\usepackage[T1]{fontenc}
\usepackage{lmodern}

\usepackage{amsmath}
\usepackage[ngerman]{babel}
\usepackage{hhline}
\usepackage{icomma}
\usepackage{qrcode}
\usepackage{textcomp}
\usepackage{tikz}
\usepackage{xcolor}
\usepackage{ulem}

\title{1. Fachschaftsvollversammlung Wintersemester 2017}
\author{Fachschaft Informatik}
\date{06.12.2017}
\subject{}

\newcommand{\matheuro}{\text{ \texteuro}}

\newcommand{\titleslide}[1]{
	\begin{frame}
		\huge #1 \normalfont
	\end{frame}
}

\usetheme{TuDoCorporate}

\begin{document}

\tudotitle

\section{Berichte der AGs}

\titleslide{\centering\includegraphics[scale=0.2]{res/logo_text-voll-transp.png}}

\begin{frame}
	\frametitle{Fest im Programm / Aktuelles}
	\begin{itemize}
		\item wöchentliche Treffen
		\item Vortragsreihen zu FOSS
		\item Linux-Installationspartys
		\item Unterstützung der O-Phase
		\item Zusammenarbeit mit lokalen Hackspaces
		\item Raspberry Pi Workshops
		\item Unterstützung des LinuxLounge Podcasts
		\item monatliches Event: Hack'n'Snack
	\end{itemize}
\end{frame}

\begin{frame}
	\frametitle{Geplant für 2018}
	\begin{itemize}
		\item Raspberry Pi Workshop an Schulen
		\item Linux Days Dortmund 2018
		\item FOSS-AG Hackathon
		\item Hack'n'Snack
		\begin{itemize}
			\item Themenabend: ARM-Prozessoren
			\item Workshop: Tensorflow
			\item ...
		\end{itemize}
		\item Ein an die Fachschaft gerichteter Podcast
	\end{itemize}
\end{frame}

\end{document}