\documentclass[xcolor=table]{beamer}

\setbeamercovered{transparent}

\usepackage[utf8]{inputenc}
\usepackage[T1]{fontenc}
\usepackage{lmodern}

\usepackage{amsmath}
\usepackage[ngerman]{babel}
\usepackage{hhline}
\usepackage{icomma}
\usepackage{qrcode}
\usepackage{textcomp}
\usepackage{tikz}
\usepackage{xcolor}
\usepackage{ulem}

\title{1. Fachschaftsvollversammlung Wintersemester 2017}
\author{Fachschaft Informatik}
\date{06.12.2017}
\subject{}

\newcommand{\matheuro}{\text{ \texteuro}}

\newcommand{\titleslide}[1]{
	\begin{frame}
		\huge #1 \normalfont
	\end{frame}
}

\usetheme{TuDoCorporate}

\begin{document}

\tudotitle

\section{Berichte der AGs}

\titleslide{\centering\includegraphics[scale=0.2]{res/logo_text-voll-transp.png}}

\begin{frame}
	\frametitle{Wann, Wo und Was?}
	\begin{itemize}
		\item Wann und Wo?
			\begin{itemize}
				\item Mittwochs 18-20 Uhr, Fachschaftsflur
			\end{itemize}
		\item Was?
		\begin{itemize}
			\item Linux-Installationspartys
			\item Vortragsreihen zu FOSS
			\item Zusammenarbeit mit lokalen Hackspaces
			\item Raspberry Pi Workshops
		\end{itemize}
	\end{itemize}
\end{frame}

\begin{frame}
	\frametitle{Weitere Infos}
	\begin{itemize}
		\item https://foss-ag.de/
		\item https://hackandsnack.de/
		\item Telegram: @FOSSAG
		\item Twitter: @FOSSAGTUDO
	\end{itemize}
\end{frame}

\end{document}